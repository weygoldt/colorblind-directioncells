Our preprocessing pipeline successfully separated the direction-selective units from the population in the optic tectum. All units we analyzed were only active in their respective assigned motion direction despite the relatively low threshold we used to select them, as shown in figure \ref{fig:heatmap}. These direction selective cells showed the lowest activity to isoluminant chromatic contrasts. The eye velocities also show this decrease, albeit not as strong as the cellular response.

\subsection{Direction selective cells in the optic tectum may be colorblind}

We found a decreased response, both on the level of the calcium activity of direction-selective tectal units - and behavior when the stimulus shifted from achromatic motion cues towards chromatic motion cues. This suggests that chromatic information is used less or not at all to encode motion direction in the optic tectum. 
% This can be compared to one of the notions of primates, where both pathways are separate \parencite{margaret1988segregationcolormovement}. 
The lack of a true zero cline and the fact that the response on the diagonal of the heatmap (figure \ref{fig:calciumdata}) still increases with increasing chromatic contrast between the stripes can be interpreted in two different ways regarding the implementation of the current version of this experiment. The first interpretation could be, that there is in fact a true zero cline on the diagonal, but the stimulus levels we used were not diverse enough to target this region. If the stimulus is never truly isoluminant, a small amount of luminance contrast would still be able to stimulate the units. This luminance contrast would not increase with the intensity of both channels, however, the overall contrast sensitivity of the visual system is dependent on light intensity. This would also explain why the response to (seemingly) isoluminant chromatic contrasts increases, as the (not purely) chromatic contrasts increase. To test this hypothesis, a stimulation setup with increased precision in its tunability would be needed. In addition to that, one would need to increase the levels of contrasts that are presented. However, since all possible combinations of contrasts are presented, the number of stimulus phases needed would increase quadratically, making this approach less realistic. The second interpretation could be, that luminance may be the primary cue that is used for motion processing, but the optic tectum receives integrated information from the retina, which contains some color information. To our knowledge, there is no data on chromatic inputs into the optic tectum that interact with motion encoding circuits \parencite{wangParallelChannelsMotion2020, ROBLES20142085}. 

% \vspace{\baselineskip}

% If we compare our results to primates, we show that zebrafish only use the first order of motion recognition. This first order relies solely on luminance and not color, indicating color blindness if we would stimulate with an isoluminant stimulus \parencite{orgerChannelingRedGreen2005, zhong1999isoluminant}. This is presented in our experiment. 

\vspace{\baselineskip}

For further studies, one could use color gratings moving in more directions than only the horizontal plane. For the zebrafish were three subtypes of direction-selective cells identified that can be used for a more robust dataset \parencite{nikolaou2012parametric}. Another suggestion can be the combination of color gratings in an optic flow stimulus, to have more natural stimuli. 

\subsection{The behavioral response is not as strong as the physiological response}

% As for the decreased response in the data from the calcium imaging, in the behavioral response, we can estimate a shift to no motion for chromatic cues. In general, the response in the behavior analysis is weaker than in the calcium data, because the zero cline is not as strongly color coded as the one in the calcium imaging. That the behavioral readout is not completely in line with our calcium data and the previous studies \parencite{orgerChannelingRedGreen2005}, can be explained by the stimulus setup. The setup is optimized for our used calcium indicator and has 4 seconds of moving stimuli and 4 seconds of stationary stimulus with one specific color grating. To improve the behavioral readout one could introduce a gap of a given time so that the eye movements of the fish do not interfere with the next presented stimuli. 

The main factor that explains elevated eye velocities, even for invisible stimuli, is our analysis pipeline. Since eye velocities are given in degrees per second, one of the stimuli will always result in negative velocities. To deal with this, we took the absolute velocities of both signals. The result is, that noise in the velocity signal that jitters around zero, and hence has a mean of approximately 0, becomes positive, with a mean above zero. This makes it hard to compare the two heatmaps in figure \ref{fig:calciumdata} and figure \ref{fig:behavdata} quantitatively. To improve on this, a future iteration of our analysis pipeline should simply switch the sign of the signal depending on the motion direction of the stimulus. This would keep the mean of noisy data around 0, while still making the velocities across the stimulus categories comparable.

\vspace{\baselineskip}

However if after this change in the pipeline, we observe the same pattern, it could be interpreted in multiple ways. The behavioral zero-cline on the heatmap of figure \ref{fig:behavdata} is not as prominent as the same figure for the calcium data in figure \ref{fig:calciumdata}. A first possible explanation could be a direct consequence of the interpretation, that a small amount of chromatic information might still be able to contribute to motion perception. In fact, more chromatic information could be integrated in areas downstream of the optic tectum, resulting in a higher behavioral sensitivity to moving chromatic cues compared to the physiological level in the optic tectum. Another explanation could be, that in contrast to \textcite{orgerChannelingRedGreen2005} we used the OKR, which could make a difference in the response to an isoluminant stimulus. But this would imply that the OKR and the OMR differ in how chromatic information is indicated in the behavioral output while at the same time sharing components in the pretectum and both encoding similar large-field motion stimuli \parencite{wang2019selective, bollmannZebrafishVisualSystem2019}. Another, perhaps easier explanation could be, that behavioral measurements are inherently noisier than measurements of neural activity. In the end, the change in the preprocessing pipeline combined with a strongly increased sampling size should suffice to explain and perhaps completely resolve this discrepancy. 

\subsection{Summary}

In conclusion, we found an indication that motion and color are processed separately in the zebrafish optic tectum and that the motion processing pathway only includes achromatic information. For the first time, we show that this may be reflected on a physiological level in the optic tectum. Additionally, we see the same pattern manifested in the behavioral output. 
