
Sensory information about the environment is crucial for survival. Especially the sensory modality of light is essential to perceive the outside world in many vertebrates. In primates, the incoming light is absorbed by three photoreceptors, which enable color vision \parencite{robert1977retinalcones}. Another visual feature processed by the same sensory modality are temporal changes such as the movement of objects. Whether these two features are processed separately and how they may still interact is strongly disputed \parencite{GEGENFURTNER1996394}. \textcite{margaret1988segregationcolormovement} argued, that the segregation of color and movement is facilitated by the pathway selectivity and differences in the encoding of "color selectivity, contrast sensitivity, temporal properties and spatial resolution" of cells in the geniculate. Cells that encode for color are in a separate pathway (parvocellular geniculate pathway, motion in magnocellular geniculate pathway) and have properties that ensure only color vision \parencite{margaret1988segregationcolormovement}. This separation of pathways from color and motion implies that motion should only be perceivable if the visual display contains luminance contrasts \parencite{MULLEN1992colormotion}. In contrast to Livingstone, \textcite{MULLEN1992colormotion} used an isoluminant chromatic stimulus to demonstrate a connection between color and motion where the motion pathway received small amounts of chromatic information and vice versa. fMRI studies with human subjects imply that color vision is physiologically widely distributed, and not defined for only one pathway \parencite{WANDELL1999fmricolor}. Motion perception can be subdivided into three orders of processing \parencite{zhong1999isoluminant}. First-order motion is the percept of objects that are defined by their luminance and is derived from a "Hassenstein-Reichardt-Detector", which consists of two units that compare the luminance at two different spatial locations \parencite{Lu:2001, BORST2011974}. Whereas second-order motion is detecting motion stimuli that consist of equal luminance but differ in "contrast, spatial frequency, texture type, or flicker" \parencite{Lu:2001}. Lastly, the third-order motion is defined as a figure relative to the background \parencite{zhong1999isoluminant}. \textcite{zhong1999isoluminant} proposed that isoluminant motion is computed by the third-order system, a brain region where "binocular inputs of form, color, depth, motion, and texture are all available" \parencite{zhong1999isoluminant}. 

From an evolutionary perspective, color vision in primates is a recent development \parencite{yokoyama2001colorvisongen}. This is not the case for zebrafish \parencite{BADEN2021R807}. To understand the physiological segregation of color- and motion perception, we explored the perception of the zebrafish (\textit{Danio rerio}), a very basal vertebrate. In fish, color vision evolved independently from primates, providing a comparative approach \parencite{nigel1967fishretina}. 

\subsection{Motion perception in zebrafish}

Zebrafish presents the unique opportunity to combine whole-brain imaging data with behavior to complex visual stimuli \parencite{bollmannZebrafishVisualSystem2019}. The retina of the zebrafish has a similar cell structure and pathways compared to mammalian retinas, making it comparable to human retinas \parencite{bollmannZebrafishVisualSystem2019}. The visual system is developed early after fertilization, in a time-span of three days \parencite{stuermer1988retinotopic}, and after the three days, zebrafish reliably respond to light \parencite{zhang2010development}. In addition to a comparable retina, zebrafish also exhibit the optokinetic response (OKR), a reflex also present in humans. The OKR describes then the movement of eyes in the direction of a moving stimulus, which is interrupted by rapid saccades in the opposite direction \parencite{brockerhoff1995behavioral}. The OKR is encoded in the velocity-to-position neural integrator (VPNI) which is located in the caudal hindbrain \parencite{miri2011regression, miri2011spatial}. Aside from the circuitry for the OKR, \textcite{perez2016sustained} results indicate that direction-selective tectal cells had a sustained rhythmic activity after stimulation that elicited the OKR. \textcite{wangParallelChannelsMotion2020} investigated receptive fields of the pretectum and tectum of motion-selective neurons. Neurons in the pretectum have large receptive fields for the lower visual fields, whereas tectal neurons have mostly small receptive fields for the upper nasal visual field \parencite{wangParallelChannelsMotion2020}. This underlines the notion that motion-selective cells in the pretectum and tectum correspond to different functions. The pretectum is vital for the optomotor response (OMR), where fish swim in the direction of the optic flow, and the tectum for near-field prey capture (\cite{wangParallelChannelsMotion2020} but see also \cite{10.3389/fncir.2021.814128}). 

\subsection{Color vision in zebrafish}

Chromatic information, on the other hand, is first absorbed by the four photoreceptors with absorption peaks around \SI{500}{\nano\meter} (red), \SI{470}{\nano\meter} (green), \SI{410}{\nano\meter} (blue), and \SI{360}{\nano\meter} (UV) in the retina \parencite{robinson1993zebrafish}. The photoreceptors are arranged in so-called "cone mosaics", and the ratios between photoreceptors in larval to adult zebrafish change \parencite{allison2010ontogeny}. The position of the receptors has evolved to match the natural environment: In  a natural scenery, there is hardly any color above the fish. Therefore the part of the retina that captures the light coming from above the fish consists is not specialized towards color vision \parencite{zimmermannZebrafishDifferentiallyProcess2018}. The majority of chromatic information can is available below the fish up to its visual horizon, which is represented on the retina by an area rich in cones \parencite{zimmermannZebrafishDifferentiallyProcess2018}. After the retina, visual stimuli are processed by the optic tectum, which is homologous to the superior colliculus, and also the pretectum and thalamus in the mammalian brain \parencite{bollmannZebrafishVisualSystem2019}. 

\vspace{\baselineskip}

The segregation of color and motion on a behavioral level in zebrafish was demonstrated by \textcite{orgerChannelingRedGreen2005}. Using a motion-nulling approach, they could show that the motion perception of zebrafish larvae relied on luminance contrast to perceive a moving stimulus \parencite{chichilniskyFunctionalSegregationColor1993}. The physiological segregation of color and motion in the zebrafish brain is still unclear. The optic tectum is one of the first areas that receive retinal input \parencite{bollmannZebrafishVisualSystem2019}, turning it into an ideal starting point to understand the physiological segregation that is reflected in the behavioral output during isoluminant chromatic motion stimulation. The question we tried to answer, was whether the tectal cells that show increases in their Calcium activity to directional motion still do so if the moving stimulus is isoluminant and the only motion cues are of chromatic nature. We investigated this question with a novel experimental setup, that can stimulate the fish globally with moving color gratings and simultaneously record  the neural Calcium activity in the brain of zebrafish larvae as well as simultaneously quantify eye movements and swimming behavior. 
