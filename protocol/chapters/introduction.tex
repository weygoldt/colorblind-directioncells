
Sensory information about our environment is crucial for our survival. Especially the sensory modality of light is essential to perceive the outside world. In primates, for example, the incoming light is perceived by three photoreceptors, and they process the chromatic information \parencite{robert1977retinalcones}. This chromatic information is letting us see the world in color. Another factor processed by the same sensory modality is temporal changes e.g the movement of an object. These two factors, color vision and movement, were thought to be functionally independent of each other. \cite{margaret1988segregationcolormovement} explained this segregation of color and movement by the pathway selectivity and differences in the encoding of \glqq color selectivity, contrast sensitivity, temporal properties and spatial resolution\grqq{} \parencite{margaret1988segregationcolormovement} of cells in the geniculate. For example, cells that encode for color are in a separate pathway (parvocellular geniculate pathway, motion in magnocellular geniculate pathway) and have properties that ensure only color vision \parencite{margaret1988segregationcolormovement}. This separation of pathways from color and motion should then indicate that motion vision is color blind \parencite{MULLEN1992colormotion}. In contrast to Livingstone, \cite{MULLEN1992colormotion} did find a connection between color and motion whereas the motion pathway has some reduced color vision and vice versa, using an isoluminant chromatic test. Further experiments with fMRI in humans also imply that color vision is widely distributed, and not defined for only one pathway \parencite{WANDELL1999fmricolor}. This system which is integrating motion sensitivity can be further subdivided into three orders of processes \parencite{zhong1999isoluminant}. The first-order is responsible for computing movements of objects that are defined by their luminance, whereas the second-order motion detection is defined by the luminance contrast and lastly the third-order motion is defined as figure relative to the background \parencite{zhong1999isoluminant}. \cite{zhong1999isoluminant} displayed that isoluminant motion is computed by the third-order system, which concludes this stimulus is computed in a brain region where \glqq binocular inputs of form, color, depth, motion, and texture are all available \grqq{} \parencite{zhong1999isoluminant}. 

\vspace{\baselineskip}

To summarize this, we tried to emphasize both viewpoints in primates. The first one is separating color and motion in two different pathways whereas the other displays a connection between the color and motion pathway. From an evolutionary perspective, color vision in primates is a recent development \parencite{yokoyama2001colorvisongen} and has influenced to some degree the motion perception in primates. To investigate the evolutionary changes in color-motion perception, we wanted to explore the perception of the zebrafish (\textit{Danio rerio}), compared to primates a very early vertebrate. In fish, the color vision evolved independently from primates, providing a comparative approach \parencite{nigel1967fishretina}. 

\subsection{Motion perception in zebrafish}

Zebrafish as model organisms presents the unique opportunity to combine whole-brain imaging data with behavior to complex visual stimuli \parencite{bollmannZebrafishVisualSystem2019}. The retina of the zebrafish has a similar cell structure and pathways compared to mammalian retinas, which makes the comparison easier \parencite{bollmannZebrafishVisualSystem2019}. The visual system is developed early after fertilization, in a time span of three days \parencite{stuermer1988retinotopic}, but after the three days, zebrafish reliably respond to light \parencite{zhang2010development}. If one would fixate on the body (leaving the eyes movable) and stimulate the fish with a whole-field motion, then it would elicit an optokinetic response (OKR). OKR describes then the movement of the eyes in the direction of the stimulus \parencite{brockerhoff1995behavioral}. This OKR is encoded in the velocity-to-position neural integrator (VPNI) which is located in the caudal hindbrain \parencite{miri2011regression, miri2011spatial}. Aside from this circuitry for the OKR, \cite{perez2016sustained} results indicate that direction-selective tectal cells had a sustained rhythmic activity after stimulation. \cite{wangParallelChannelsMotion2020} investigated receptive fields of the pretectum and tectum of motion-selective neurons. Neurons in the pretectum have large receptive fields for the lower visual fields, whereas tectal neurons have mostly small receptive fields for the upper nasal visual field \parencite{wangParallelChannelsMotion2020}. This underlines the notion that motion-selective cells in the pretectum and tectum correspond to different functions. The pretectum is important for the optomotor response (OMR), where fish swim in the direction of the optic flow, and the tectum for near-field prey capture \parencite{wangParallelChannelsMotion2020}. 

\subsection{Color vision in zebrafish}

Color vision, on the other hand, is first perceived in the four photoreceptors with absorption peaks around \SI{500}{\nano\meter} (red), \SI{470}{\nano\meter} (green), \SI{410}{\nano\meter} (blue), and \SI{360}{\nano\meter} (UV) \parencite{robinson1993zebrafish}. The photoreceptors are arranged in a pattern which is called \glqq cone mosaics\grqq{}, and the ratios between photoreceptors in larval to adult zebrafish change \parencite{allison2010ontogeny}. The proportions of the photoreceptors change in the development but the position of the receptors is evolved to fit for the natural stimuli, above the fish, in  a natural scenery are hardly any color. Therefore the retina encoding for light trajectories coming from above the fish is primarily achromatic \parencite{zimmermannZebrafishDifferentiallyProcess2018}. More color is found beneath the fish and the horizontal level, which is represented in the retina \parencite{zimmermannZebrafishDifferentiallyProcess2018}. After the retina visual stimuli are processed by the optic tectum, if compared to a mammalian structure, is homologous to the superior colliculus, and also the pretectum and thalamus \parencite{bollmannZebrafishVisualSystem2019}. 

\subsection{Aim of this work}
The first seperation of color and motion in zebrafish was shown behaviorally by \cite{orgerChannelingRedGreen2005}. They could show that zebrafish larvae are motion blind to a chromatic (isoluminant) stimulus, using a motion nulling method \parencite{chichilniskyFunctionalSegregationColor1993}. The physiological segregation of color and motion in the zebrafish brain is still unclear.  We investigated this question with a new setup, that can stimulate the fish globally with color gratings and simultaneously record  CA imaging data of the optic tectum and pretectum with a 2-photon infrared laser. 


% \begin{itemize}

% \item Early on, photoreceptors
% sort by wavelength (Marks et al., 1964; Marc \&  Sperling, 1977),
% while bipolar cells sort increments and decrements of light (Werblin \& Dowling, 1969).
% \href{https://www.cambridge.org/core/journals/visual-neuroscience/article/channeling-of-red-and-green-cone-inputs-to-the-zebrafish-optomotor-response/180B0E41BDD8D237586D42D07B9908E0}{oger2005}

% \item Stimulus color and contrast influence perceived stimulus speed. For example, a stimulus can be made to appear to move at a different rate by adjusting its color or contrast (Cavanagh et al. 1984). \href{https://doi.org/10.1016/S0896-6273(00)81037-5}{wandall1999}

% \item "in vision, specializations are often
% made according to the statistics of specific regions in visual
% space. For example, mouse cones preferentially process dark
% contrasts above, but not below, the visual horizon, likely boosting the detection of aerial predators [3, 4]."  
% \href{https://doi.org/10.1016/j.cub.2018.04.075}{wangarrenberg2020motion-extraction}



% \item \textbf{Vision in zebrafish}
% \item Zebrafish larvae perform a wide range of visually mediated behaviours, ranging from prey capture
% (Trivedi and Bollmann, 2013; Mearns et al., 2020) and escape behaviour (Heap et al., 2018) to stabilisation behaviour (Kubo et al., 2014; Orger et al., 2008); however, the importance of stimulus
% location within the visual field for the execution of the respective behaviors has only recently been
% recognized and is still not well understood (Hoy et al., 2016; Zimmermann et al., 2018;
% Mearns et al., 2020; Kist and Portugues, 2019; Wang et al., 2020; Johnson et al., 2020;
% Lagogiannis et al., 2020)
% \href{https://doi.org/10.7554/eLife.63355}{Dhemelt2021stimulus-location}

% \item Within three days of hatching, larval zebrafish become highly
% visual animals with tetrachromatic wide-angle vision [9, 10] and
% well-studied visual behaviors [11–16]. 
% \href{https://doi.org/10.1016/j.cub.2018.04.075}{wangarrenberg2020}

% \item the two eyes make up nearly a quarter
% of the total body volume, with the neuronal retina taking up $>$75\%
% of each eye. Indeed, about half of the larva’s central neurons are
% located inside the eyes (STAR Methods). 
% \href{https://doi.org/10.1016/j.cub.2018.04.075}{wangarrenberg2020}

% \item \textbf{our experiment}
% \item motion-nulling method \href{https://doi.org/10.1016/0042-6989(93)90010-T}{chichilnisky1993motion-nulling}
% \end{itemize}